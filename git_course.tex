\documentclass{beamer}
\usetheme[numbering=none,nofirafonts]{gitcourse}

\definecolor{main}{RGB}{92, 138, 168}
\definecolor{background}{RGB}{240, 247, 255}

\usepackage{sourcecodepro}
%\usepackage[osf]{mathpazo}

\usepackage{booktabs}

\newcommand{\git}{Git{}}

\title{\git, GitHub and GitFlow}
\author{Jacopo Schiavon}
\institute[Dep. Statistical Sciences --- UNIPD]{Department of Statistical Sciences\\ University of Padova}


\begin{document}
\begin{frame}[plain]
    \maketitle
\end{frame}

\begin{frame}{}
\tableofcontents
\end{frame}

\section{Introduction to \git}
\begin{frame}{What is \git?}
    \alert{\git} is a \alert{free} and \alert{open source} distributed \alert{version control system} designed to handle everything from small to very large projects with speed and efficiency. 
    
    Its goals include \alert{speed}, \alert{data integrity}, and support for \alert{distributed}, \alert{non-linear workflows}.
    
    \begin{block}{History of \git}
        \begin{itemize}
            \item \git{} was created in 2005 by Linus Torvald (now maintained by Junio Hamano)
            \item Its main purpose was helping developing the Linux kernel
            \item It is designated for coordinating the work of many developer on the same project
            \item It is very useful also for smaller projects
        \end{itemize}
    \end{block}
\end{frame}

\begin{frame}{Why \git?}
    \git{} is extremely useful when managing \alert{projects consisting of text files}, such as \texttt{R} or \texttt{Python} packages and scripts, \LaTeX{} documents or text files repositories (such as configuration files for servers or desktops).
    
    It gives its best when it is used cooperatively by a \alert{team of developers} that can exploit its most advanced functions, but is useful also for smaller projects:
    \begin{itemize}
        \item It teaches \alert{good practices} in code developing and maintaining
        \item It greatly simplify \alert{versioning} and adding experimental or new feature to the code
        \item Combined with GitHub, it trivialize the \alert{code sharing} procedure
        \item It helps with keeping track of relations between files and the \alert{history of the code}.
    \end{itemize}
\end{frame}

\begin{frame}{\git{} features}
    \begin{itemize}
        \item \alert{Branching} \git{} main feature, will have an entirely dedicated slide.
        \item \alert{Small and Fast} All \git{} procedures are fast and minimal: one or two commands to do everything. Committing and pushing requires no more than a couple of seconds $\Rightarrow$ almost no impact on the workflow.
        \item \alert{Entire history of the repository} \git{} keeps all versions of all files that it is tracking. Allows you to check differences between every two versions (or reverting to any version) of your code.
        \item \alert{Distribution} \git{} structure implies that each developer works on its own (forked) version of a central repository, without breaking the work of others developers.
        \item \alert{Cryptographic assurance} Every bit that \git{} tracks is authenticated by advanced cryptographic techniques.
    \end{itemize}
\end{frame}

\begin{frame}{\git{} features: Branching}
    \begin{itemize}
    \item The main idea that led to developing \git{} is the possibility to \alert{work in parallel} on the same code. This is most useful when multiple developers work at the same project, but also when only one works on it.
    
    \item This importance is due to the possibility of \alert{developing different feature} of the code without the need of creating new files or appending useless line to the code. 
    
    \item \git{} allows to \alert{create}, \alert{switch} and \alert{merge} different branches with minimal effort (one command for each operation).
    \end{itemize}
    \begin{block}{Examples of use cases}
        A different implementation of a particular function we are still unsure will work better, rewriting a paragraph or translating the entire document in another language, adding a new functionality that might compromise other parts of the code.
    \end{block}
\end{frame}

\begin{frame}{Some commands}
\begin{center}
\footnotesize
\begin{tabular}{p{0.3\textwidth}p{0.6\textwidth}}
    \toprule
    \alert{Command}	&	\alert{Meaning}\\
    \midrule
    \texttt{git add \textit{file}}	&	Add \texttt{\textit{file}} to the staging area\\
    \texttt{git add .}	&	Add all the modified files in the repository to the staging area\\
    \texttt{git commit}	&	Create a snap of the staged files\\
    \texttt{git commit -m \textit{""}} & Add a message to the commit\\
    \texttt{git commit -a} & Commit all the tracked and modified files\\
    \texttt{git push origin \textit{master}}	& Pushes the branch commits to the \textit{master} branch of online repository origin\\
    \bottomrule
\end{tabular}
\end{center}
\end{frame}

\section{Introduction to GitHub}
\begin{frame}{Frame Title}
\end{frame}

\section{Introduction to GitFlow}
\begin{frame}{Frame Title}
\end{frame}



\end{document}


